%*************************************************
\documentclass[a4paper,10pt]{article}
\usepackage[brazil]{babel}
\usepackage[utf8]{inputenc}
%codigo
%\usepackage[latin1]{inputenc}
\usepackage{amsthm,amsfonts,amsmath,amssymb}
%img
\usepackage{graphicx}
\usepackage{subfig}
%tabela

%fim
%fim
\usepackage{listings}
\usepackage{makeidx}
\usepackage{enumerate}
\usepackage{hyperref}
\hypersetup{
  colorlinks,
  linkcolor=blue,
  filecolor=blue,
  urlcolor=blue,
  citecolor=blue
}

%titulo
\title{Curriculum Vitae}
\author{Leandro Kümmel Tria Mendes}
%inicio
\begin{document}
\begin{center} 
Curriculum Vitae
\end{center} 
\emph{Nome}: Leandro Kümmel Tria Mendes\\
\emph{Contato}: leandro.kummel@gmail.com | (55) (11) 99721-7809\\
\emph{Idade}: 29 anos (24/04/1986)\\
\emph{Nacionalidade}: Brasileiro\\
\section{Resumo}
Possui sólido conhecimento em desenvolvimento, principalmente voltado à internet, com mais de 11 anos de experiência em diversos projetos que vão desde, um simples \emph{Gerenciador de Patentes} a um complexo \emph{Sistema de Ensino Adaptativo}. Experiências em [AQUI VAO AS CARACTERÍSTICAS TÉCNICAS DE ACORDO COM A VAGA] PHP, HTML/CSS, JavaScript (frameworks JQuery[UI], ExtJS e PrototypeJS), HTML5, C, Java, Python, MySQL, PostgreSQL e Linux.
\section{Objetivo}
Foco em estudos de data mining, aprendizado (não-)supervisionado, processamento de imagens e desenvolvimento web. [OBJETIVOS TÉCNICOS DE ACORDO COM A VAGA]
\section{Formação}
\begin{itemize}
\item \emph{Ensino Superior}: Graduado em Engenharia de Computação Unicamp 
\item \emph{Ensino Médio}: Colégio Bandeirantes

\end{itemize}
\section{Experiências profissionais [VARIAR DE ACORDO COM VAGA]}
\begin{itemize}

\item \emph{Clickideia Tecnologia Educacional}: 
  \subitem - Responsável pela gestão dos projetos em TI .
  \subitem - Domínio de todas as fases de um projeto (iniciação, planejamento, execução, monitoramento/controle e encerramento). Para as soluções de software utilizou-se o desenvolvimento ágil, \textbf{Scrum}, na fase de execução dos projetos. Já o restante a base foi o \textbf{PMBOK}.
  \subitem - Encarregado pela montagem/contratação de equipes técnicas.
  \subitem - Responsável pela reformulação do departamento de TI, subdividindo-o em três pilares os quais passaram a ser monitorados a partir de indicadores (KPI) e objetivando equipes de alto desempenho. Além disso, diversos processos existentes foram reformulados a partir da observação de melhores práticas e resultados.
  \subitem - Intendente pela elaboração de documentos técnicos para licitações públicas
  \subitem \emph{Principal Projeto}: Portal 3.0. Portal Web de educação a distância com ensino adaptativo (inteligência artificial) com base em análise da big data (data mining).
  
\item \emph{Domane consultoria (Jan/2010 - Jan/2013) PJ }: Diversos projetos, todos web. O mais expressivo, SiGe (Sistema Integrado de Gestão Empresarial), faz coleta de dados da empresa, apresenta relatórios tanto em tabela quanto gráfico (inclusive curva ABC), e ajuda na tomada de decisão por parte do dirigente. Para o desenvolvimento da leiaute do SiGe utilizou-se HTML5 + JavaScript (Jquery), principalmente na implementação dos relatórios.Linguagens \textbf{HTML/CSS, HTML5, PHP5, JavaScript (frameworks: Jquery [JSon], JqueryUI, ExtJS) PostgreSQL}
\item \emph{Coordenadoria de Projetos e Obras Unicamp (Maio/2009 - Jan/2010) Bolsista}: Responsável pela administração do SigPOd (Sistema Integrado de Projetos e Obras) e desenvolvimento de novos módulos, inclusive para mobiles/tablets (introdução do HTML5 para alguns).Linguagens \textbf{HTML/CSS, HTML5, PHP5, JavaScript (frameworks: Jquery [JSon], JqueryUI), MySQL}
  
\item \emph{Instituto de Biologia Unicamp (Ago/2008 - Fev/2009) Bolsista }: Projeto de Ensino a Distância, desenvolvendo, principalmente os leiautes de cursos a distância para a rede pública do Ensino Médio, ambiente de desenvolvimento Linux, produção MacOs. Linguagens \textbf{HTML/CSS, PHP5, JavaScript (frameworks: Jquery[JSon], JqueryUI) e PostgreSQL}

\item \emph{Griffon (Dez/2006 - Nov/2008) CLT e PJ }: De Dez/2006 à Jun/2007 foi resposável pelo desenvolvimento de alguns sistemas simples, tal como Intranet e portal de comunicação entre cliente e atendimento da empresa (incluindo sistema de chat). De Jun/2007 a Nov/2008 foi promovido a Diretor de Tecnologia, gerenciando projetos de grande complexidade, tal como a leitura e recorte das noticias publicadas no Diário Oficial (tanto da União quanto Estaduais), e a classificação das mesmas. Diferentes linguagens foram utilizadas, \textbf{PHP5, HTML/CSS, AJAX(JavaScript + XML), MySQL} para os projetos web, e \textbf{Java, Python (scikit-learn) e PostgreSQL} para o projetos de leitura e classificação das notícias. Sistema operacional de desenvolvimente e produção, Linux.

\item \emph{Mandic (Jan/2006 - Dez/2006) PJ }: Desenvolvedor e coordenador do projeto WebMail 2.0. Projeto desenvolvido em \textbf{HTML/CSS, PHP5, PostgreSQL, AJAX(JavaScript + XML)}, ambiente Linux.
  
\item \emph{Inova Unicamp (Junho/2004 - Dez/2005) Bolsista }: Desenvolveu e administrou os sites da Agência de Inovação da Unicamp e da Incubadoras de empresas (Incamp) \textbf{[PHP4, MySQL, HTML/CSS e JavaScript]}, servidor de produção Windows e ambiente de desenvolvimento Linux.

\end{itemize}
\section{Informações adicionais}
\begin{itemize}
\item \emph{Língua}: Inglês fluente, vivência de 4 meses na Nova Zelândia Dez/2008 à Mar/2009.
\end{itemize}
\end{document} 

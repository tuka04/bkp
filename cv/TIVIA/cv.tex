%*************************************************
\documentclass[a4paper,10pt]{article}
\usepackage[brazil]{babel}
\usepackage[utf8]{inputenc}
%codigo
%\usepackage[latin1]{inputenc}
\usepackage{amsthm,amsfonts,amsmath,amssymb}
%img
\usepackage{graphicx}
\usepackage{subfig}
%tabela

%fim
%fim
\usepackage{listings}
\usepackage{makeidx}
\usepackage{enumerate}
\usepackage{hyperref}
\hypersetup{
  colorlinks,
  linkcolor=blue,
  filecolor=blue,
  urlcolor=blue,
  citecolor=blue
}

%titulo
\title{Curriculum Vitae}
\author{Leandro Kümmel Tria Mendes}
%inicio
\begin{document}
\begin{center} 
Curriculum Vitae
\end{center} 
\emph{Nome}: Leandro Kümmel Tria Mendes\\
\emph{Contato}: leandro.kummel@gmail.com | (55) (11) 99721-7809\\
\emph{Idade}: 29 anos (24/04/1986)\\
\emph{Nacionalidade}: Brasileiro\\
\section{Resumo}
\begin{itemize}
\item Engenheiro da computação graduado pela Unicamp, com 11 anos de atuação na área de TI (sendo 6 como gestor de projetos), possui experiências em desenvolvimento, análise, engenharia de software, manuntenção e coordenação de equipes e gestão de projetos e processos;
\item Experiências plena de 6 anos com guia PMBOK (PMI) e desenvolvimento ágil Scrum. Ferramentas MSProject, Redmine, Jira e ProjectLibre;
\item Domínio de todas as fases de um projeto (iniciação, planejamento, execução, monitoramento/controle e encerramento);
\item Conhecimento em análise de negócio é significativo;
\item Experiência plena com práticas ITIL e COBIT.
\item Experiências técnicas senior com as linguagens JAVA, PHP, Ruby, HTML/CSS, JavaScript, Python, MySQL, SQL Server, PostgreSQL, Windows e Linux;
\end{itemize}
\section{Objetivo}

  Foco em gestão de projetos, objetivando a formação de equipes de alto desempenho, além da constante colaboração no processo de ensino e aprendizem em equipe. 
\section{Formação}
\begin{itemize}
\item \emph{Ensino Superior}: Graduado em Engenharia de Computação Unicamp 
\item \emph{Ensino Médio}: Colégio Bandeirantes

\end{itemize}
\section{Experiências profissionais}
\begin{itemize}

\item \emph{Clickideia Tecnologia Educacional}: 
  \subitem - Responsável pela gestão dos projetos e operação dos processos em TI.
  \subitem - Para as soluções de software utilizou-se o desenvolvimento ágil, \textbf{Scrum}, na fase de execução dos projetos. Já o restante a base utilizada foi o guia \textbf{PMBOK} e a principal ferramenta de auxílio \textbf{MSProject}.
  \subitem - Responsável pela reformulação do departamento de TI, padronização e criação de processos apoiado pelo ITIL.
  \subitem - Encarregado pela montagem/contratação de equipes técnicas.
  \subitem - Intendente pela elaboração de documentos técnicos para licitações públicas.
  \subitem \emph{Principal Projeto}: Portal 3.0. Portal Web de educação a distância com ensino adaptativo (inteligência artificial) com base em análise da big data (BI - ETL).

\item \emph{Domane consultoria (Jan/2010 - Set/2013) PJ }: 
  \subitem - Responsável pela gestão técnica dos projetos em TI. 
  \subitem - Uso do desenvolvimento ágil \textbf{Scrum} utilizando o \textbf{Redmine} como ferramenta de apoio. Responsável pela coleta de requisitos, desenvolvimento do escopo e cronograma (\textit{sprint}), escalonamento e delegação das tarefas, monitoramento/controle de qualidade.
  \subitem - Intendente pela elaboração de relatórios técnicos para as áreas estratégica e operacional.
  \subitem \emph{Principal Projeto}:  SiGe, Sistema Integrado de Gestão Empresarial, coleta de dados operacionais, apresentação de relatórios e indicações para o auxílio na tomada de decisões pelos gestores e dirigentes.

\item \emph{Coordenadoria de Projetos e Obras Unicamp (Maio/2009 - Jan/2010) Bolsista}: 
  \subitem - Responsável pela administração técnica do SigPOd (Sistema Integrado de Projetos e Obras)
  \subitem - Desenvolvedor (Web) de novos módulos, com \textit{layout} responsivo.
  \subitem - Principais linguagens \textbf{HTML/CSS, HTML5, JAVA (JSP, JSE), JavaScript (frameworks: Jquery [JSon], JqueryUI), MySQL}, ambiente Windows (servidores e estações).
  
\item \emph{Instituto de Biologia Unicamp (Ago/2008 - Fev/2009) Bolsista }: 
  \subitem - Desenvolvedor (Web) de módulos para sistemas de ensino a distância para a rede pública do ensino médio.
  \subitem - Linguagens \textbf{HTML/CSS, PHP5, JavaScript (frameworks: Jquery[JSon], JqueryUI) e PostgreSQL}, ambiente linux.

\item \emph{Griffon (Dez/2006 - Nov/2008) CLT e PJ }: 
  \subitem - Dez/2006 à Jun/2007 foi resposável pelo desenvolvimento de sistemas, intranet e portal de comunicação entre cliente e atendimento da empresa. 
  \subitem - Linguagens \textbf{Python (skit-learn), PHP5, HTML/CSS, AJAX(JavaScript + XML), MySQL}, ambiente linux.
  \subitem - De Jun/2007 a Nov/2008 foi promovido a Diretor de Tecnologia, focado principalmente na gestão do projeto para a leitura, recorte e classificação das noticias publicadas nos Diários Oficiais. 

\end{itemize}
\section{Informações adicionais}
\begin{itemize}
\item \emph{Língua}: Inglês fluente, vivência de 4 meses na Nova Zelândia Dez/2008 à Mar/2009.
\end{itemize}
\end{document} 
